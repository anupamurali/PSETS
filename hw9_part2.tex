\documentclass[12pt]{article}

%%%%%%%%%%%%%%%%%%%%%%%%%%%%%%%%%%%%%%%%%%%%%%%%%%%%%%%%
% Packages go below
%%%%%%%%%%%%%%%%%%%%%%%%%%%%%%%%%%%%%%%%%%%%%%%%%%%%%%%%
%
\usepackage{fullpage}
\usepackage{graphicx}

\usepackage{amssymb}
\usepackage{amsthm}
\usepackage{amsmath}
\usepackage{multicol}
\usepackage{asymptote}
\usepackage{epstopdf}
\DeclareGraphicsRule{.tif}{png}{.png}{`convert #1 `dirname #1`/`basename #1 .tif`.png}

\newcommand{\problem}[1]{\vspace{0.3in} \noindent {\bf Problem #1}}
\newcommand{\solution}[1]{\vspace{0.3in} \noindent\bf Solution #1}

%%%%%%%%%%%%%%%%%%%%%%%%%%%%%%%%%%%%%%%%%%%%%%%%%%%%%%%%
% Assignment specific titles below
%%%%%%%%%%%%%%%%%%%%%%%%%%%%%%%%%%%%%%%%%%%%%%%%%%%%%%%%
%
\title{\bf \large Harvard University \\ Math 25A \\ \vspace{0.2in} Problem Set 9}
\author{ \bf \large Problem set by  Anupa Murali}
\date{\today}                

\begin{document}
\maketitle

\thispagestyle{empty}



\vspace{0.35in}

\centerline{\bf \large Part 2}
\problem{1} From the spectral theorem, we know that if $T$ is self adjoint, then there exists an orthonormal basis of $V$ consisting of eigenvectors of $T$. So $v$ can be written as follows,
$$
v = \langle v, e_1 \rangle e_1 + \cdots + \label v, e_n \rangle e_n.
$$
Applying $T$ to the above,
$$
Tv = \lambda_1\langle v, e_1 \rangle e_1 + \cdots + \lambda_n\langle v, e_n \rangle e_n.
$$
We are given that $||Tv - \lambda v || < \epsilon$, so we have,
\begin{eqnarray*}
||Tv - \lambda v ||^2 &=& ||(\lambda_1 - \lambda)\langle v, e_1 \rangle e_1 + \cdots + (\lambda_n- \lambda)\langle v, e_n \rangle e_n||^2\\
&=& |\lambda_1 - \lambda|^2|\langle v, e_1 \rangle|^2 + \cdots + |\lambda_n- \lambda|^2|\langle v, e_n \rangle|^2\\
&\ge& (\text{min} \{|\lambda_1 - \lambda|^2, \ldots, |\lambda_n - \lambda|^2\}(|\langle v, e_1 \rangle|^2 + \cdots + |\langle v, e_n \rangle|^2)\\
&=& \text{min}\{|\lambda_1 - \lambda|^2, \ldots, |\lambda_n - \lambda|^2\},
\end{eqnarray*}
so $||Tv - \lambda v ||^2 \ge \text{min}\{|\lambda_1 - \lambda|^2, \ldots, |\lambda_n - \lambda|^2\}$. But $\epsilon > ||Tv - \lambda v ||^2$, so for some eigenvalue $\lambda'$ of $T$, $|\lambda - \lambda'| < \epsilon$.

\problem{2} Suppose $\{v_1, \ldots, v_n\}$ is a basis of $V$ such that $\{v_1, \ldots, v_n\}$ are eigenvectors of $T$ and $\{\lambda_1, \ldots, \lambda_n\}$ are the eigenvalues of $T$. Then $\mathcal{M}(T^2)$ with respect to $ \{v_1, \ldots, v_n\}$ is the following,
$$
\left(
 \begin{array}{ccccc}
   \lambda_1^2 & && &0 \\
    &\cdot & & &\\
    & & \cdot& &\\
    & & & \cdot&\\
    0& & & &\lambda_n^2 \\
 \end{array}
\right),\\
$$
so $\text{trace}(T^2) = \lambda_1^2 + \cdots + \lambda_n^2 \ge 0$.\\
\qed

\problem{3} If $\langle S, T \rangle = \text{trace}(ST^*)$ defines an inner product on $\mathcal{L}(V)$, then we must show that it satisfies positivity, definiteness, additivity in the first slot, homogeneity in the first slot, and conjugate symmetry.\\

\noindent
Let $S = T^*$. We know that $\langle T, T \rangle = \text{trace}(TT^*) = \text{trace}(S^*S)$. Let $(v_1, \ldots, v_n)$ be an orthonormal basis of $V$. Then we have the following,
\begin{eqnarray*}
\text{trace}(S^*S) &=& \langle S^*Sv_1, v_1 \rangle + \cdots + \langle S^*Sv_n, v_n \rangle\\
&=& \langle Sv_1, Sv_1 \rangle + \cdots + \langle Sv_n, Sv_n \rangle\\
&=& ||Sv_1||^2 + \cdots + ||Sv_n||^2\\
\ge 0,
\end{eqnarray*}
so $\langle T, T \rangle \ge 0$. \\

\noindent
Now, suppose $T \in V$ such that $\langle T, T \rangle = \text{trace}(TT^*) =0$. for any $v \in V$, however, note that,
\begin{eqnarray*}
\langle T^*Tv, v \rangle &=& \langle Tv, Tv \rangle\\
&\ge& 0,
\end{eqnarray*}
so $T^*T$ is positive for every $v \in V$. This means that $\text{trace }T^*T \ge 0$. But $\text{trace}(TT^*) =0$, so it must be that $T = T^* = 0$. \\

\noindent
If $T = T^* = 0$, we know that $TT^*$ is a 0 matrix, so trace $(TT^*) = 0$. \\

\noindent
 Now, let $S_1, S_2, T \in \mathcal{L}(V)$. We have the following,
 \begin{eqnarray*}
 \langle S_1, T \rangle &=& \text{trace}(S_1T^*)\\
 \langle S_2, T \rangle &=& \text{trace}(S_2T^*)\\
 \Longrightarrow \langle S_1, T \rangle + \langle S_2, T \rangle &=& \text{trace}(S_1T^*) +  \text{trace}(S_2T^*)\\
 &=& \text{trace}(S_1T^* + S_2T^*)\\
 &=& \text{trace}((S_1 + S_2)T^*)\\
 &=& \langle S_1 + S_2, T \rangle.
 \end{eqnarray*} 

 
 \noindent
 Now let $c \in \mathbb{F}$, so we have,
 \begin{eqnarray}
 \langle cS_1, T \rangle &=& \text{trace}(cS_1T^*) \nonumber \\
 &=& c \text{trace}(S_1T^*)  \\
 &=& c \langle S_1, T \rangle. \nonumber
 \end{eqnarray}
We know that (1) holds true since trace$(T)$ is the sum of the entries in the diagonal of $\mathcal{M}(T)$. $\mathcal{M}(cT)$ contains the entries in $\mathcal{M}(T)$ each multiplied by $c$, so $\text{trace}(cT) = c\text{trace }T$, so $\text{trace}(cS_1T^*) = c \text{trace}(S_1T^*) $.\\


\noindent
Let $(v_1, \ldots, v_n)$ be an orthonormal basis for $V$ and $S \in \mathcal{L}(V)$. Then we have the following,
\begin{eqnarray*}
\text{trace }(ST^*) &=& \langle ST^*v_1, v_1 \rangle + \cdots + \langle ST^*v_n, v_n \rangle\\
&=& \langle v_1, ST^*v_1 \rangle + \cdots + \langle v_n, ST^*v_n \rangle\\
&=& \overline{\langle (ST^*)^*v_1, v_1 \rangle} + \cdots +\overline{\langle (ST^*)^*v_n, v_n \rangle}\\
&=& \overline{\text{trace}((ST^*)^*)},
\end{eqnarray*}
So,
\begin{eqnarray*}
\langle S, T \rangle &=& \text{trace}(ST^*)\\
&=& \overline{\text{trace}((ST^*)^*)}\\
&=& \overline{\text{trace}(TS^*)}\\
&=& \overline{\langle S, T \rangle}.
\end{eqnarray*}
So $\langle S, T \rangle = \text{trace}(ST^*)$ satisfies definiteness, additivity in the first slot, homogeneity in the first slot, and conjugate symmetry, so it defines an inner product.\\
\qed
\end{document}