\documentclass[12pt]{article}

%%%%%%%%%%%%%%%%%%%%%%%%%%%%%%%%%%%%%%%%%%%%%%%%%%%%%%%%
% Packages go below
%%%%%%%%%%%%%%%%%%%%%%%%%%%%%%%%%%%%%%%%%%%%%%%%%%%%%%%%
%
\usepackage{fullpage}
\usepackage{graphicx}

\usepackage{amssymb}
\usepackage{amsthm}
\usepackage{amsmath}
\usepackage{multicol}
\usepackage{asymptote}
\usepackage{epstopdf}
\DeclareGraphicsRule{.tif}{png}{.png}{`convert #1 `dirname #1`/`basename #1 .tif`.png}

\newcommand{\problem}[1]{\vspace{0.3in} \noindent {\bf Problem #1}}
\newcommand{\solution}[1]{\vspace{0.3in} \noindent\bf Solution #1}

%%%%%%%%%%%%%%%%%%%%%%%%%%%%%%%%%%%%%%%%%%%%%%%%%%%%%%%%
% Assignment specific titles below
%%%%%%%%%%%%%%%%%%%%%%%%%%%%%%%%%%%%%%%%%%%%%%%%%%%%%%%%
%
\title{\bf \large Harvard University \\ Math 25B \\ \vspace{0.2in} Problem Set 9}
\author{ \bf \large Problem set by  Anupa Murali}
\date{\today}                

\begin{document}
\maketitle

\thispagestyle{empty}



\vspace{0.35in}
\centerline{\bf \large Part 1}

\problem{1} Note that $\varphi$ is surjective, since $\varphi$ takes $[a,b] \rightarrow [c,d]$. Further, it is injective, since $\varphi$ is strictly increasing. Hence $\varphi$ is bijective, so it is invertible. So if $\alpha = \beta \circ \varphi$, then $\varphi^{-1} \circ \alpha = \beta$. For $\varphi$ to be symmetric, then we must have that $\varphi^{-1}>0$. Note that,
\begin{eqnarray*}
\varphi \circ \varphi^{-1} &=& I\\
\Longrightarrow \varphi ' (\varphi^{-1}) \cdot {\varphi'^{-1}} &=& 1\\
\Longrightarrow \varphi ' (\varphi^{-1}) &=& \frac{1}{\varphi'^{-1}},
\end{eqnarray*}
but since $\varphi'(\varphi^{-1})>0$, $\varphi'^{-1}>0$. Hence if $\alpha$ is equivalent to $\beta$, $\beta$ is equivalent to $\alpha$.\\
\qed


\noindent
Now, suppose that $\alpha = \beta \circ \varphi$ and $\beta = \gamma \circ f$. So we have the following,\\
\begin{eqnarray*}
\alpha &=& \gamma \circ f(\varphi).
\end{eqnarray*}
Hence we must show that $f'(\varphi)\varphi' > 0$. Note that $\varphi' > 0$, since $\alpha$ is equivalent to $\beta$. Further, $f'>0$, since $\beta$ is equivalent to $\gamma$. From this, we know that $f'(\varphi)\varphi'>0$.\\
\qed


\problem{2} If $\alpha'(t) \ne 0,~ \forall t$, then we have the following,
\begin{eqnarray*}
\beta'(t) &=& \alpha'(\varphi(t)) \varphi'(t) \ne 0.
\end{eqnarray*}
Since $\varphi'(t)>0$, it must be that $\beta'(t) \ne 0$ as well. Hence if $\alpha$ is smooth, $\beta$ is also smooth.\\
\qed

\end{document}