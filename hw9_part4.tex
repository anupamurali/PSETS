\documentclass[12pt]{article}

%%%%%%%%%%%%%%%%%%%%%%%%%%%%%%%%%%%%%%%%%%%%%%%%%%%%%%%%
% Packages go below
%%%%%%%%%%%%%%%%%%%%%%%%%%%%%%%%%%%%%%%%%%%%%%%%%%%%%%%%
%
\usepackage{fullpage}
\usepackage{graphicx}

\usepackage{amssymb}
\usepackage{amsthm}
\usepackage{amsmath}
\usepackage{multicol}
\usepackage{asymptote}
\usepackage{epstopdf}
\DeclareGraphicsRule{.tif}{png}{.png}{`convert #1 `dirname #1`/`basename #1 .tif`.png}

\newcommand{\problem}[1]{\vspace{0.3in} \noindent {\bf Problem #1}}
\newcommand{\solution}[1]{\vspace{0.3in} \noindent\bf Solution #1}

%%%%%%%%%%%%%%%%%%%%%%%%%%%%%%%%%%%%%%%%%%%%%%%%%%%%%%%%
% Assignment specific titles below
%%%%%%%%%%%%%%%%%%%%%%%%%%%%%%%%%%%%%%%%%%%%%%%%%%%%%%%%
%
\title{\bf \large Harvard University \\ Math 25A \\ \vspace{0.2in} Problem Set 9}
\author{ \bf \large Problem set by  Anupa Murali}
\date{\today}                

\begin{document}
\maketitle

\thispagestyle{empty}



\vspace{0.35in}

\centerline{\bf \large Part 4}
\problem{1} The sum of the diagonals of the matrix is the sum of the eigenvalues of $T$. Hence the sum of the eigenvalues of $T$ is $51 - 40 + 1 = 12$, so the last eigenvalue must be $12 - (-48 + 24) = \boxed{36}$.\\

\problem{2} First, suppose that there is an inner product on $U$ that makes $T$ into a self-adjoint operator. Then by the spectral theorem, $U$ has an orthonormal basis consisting of eigenvectors of $T$.\\

\noindent
Now  Suppose $U$ has a basis $(v_1, \ldots, v_n)$ of eigenvectors of $T$, so every element of $U$ can be written as unique linear combination of $(v_1, \ldots, v_n)$. If $u, v \in U$ such that $u = a_1v_1 + \cdots + a_nv_n$ and $v = b_1v_1 + \cdots + b_nv_n$, then we define $\langle u, v \rangle$ as follows,
$$
\langle u, v \rangle = a_1b_1 + \cdots + a_nb_n.
$$
We first show that such a definition is indeed an inner product. First, we have,
$$
\langle u, u \rangle = a_1^1 + \cdots + a_n^2 \ge 0,
$$
so the definition satisfies positivity.\\

\noindent
Next suppose $\langle u, u \rangle = 0$. Then we have, 
$$
\langle u, u \rangle =  a_1^2 + \cdots + a_n^2 = 0,
$$
so $a_1 = a_2 = \cdots = a_n = 0$, so $u = 0$.\\

\noindent
If $u, v, w \in U$, then we have, where $u = a_1v_1 + \cdots + a_nv_n$, $v = b_1v_1 + \cdots + b_nv_n$, and $w = c_1v_1 + \cdots + c_nv_n$,\\
\begin{eqnarray*}
\langle u, v \rangle &=& a_1b_1 + \cdots + a_nb_n\\
\langle u, w \rangle &=& a_1c_1 + \cdots + a_nc_n.
\end{eqnarray*}
Since $v + w = (b_1 + c_1)v_1 + \cdots + (b_n + c_n)v_n$, we have,
\begin{eqnarray*}
\langle u, v \rangle + \langle u, w \rangle = &=& a_1(b_1 + c_1) + \cdots + a_n(b_n + c_n)\\ 
&=& \langle u, v + w \rangle,
\end{eqnarray*} 
so the definition also satisfies additivity in the first slot.\\

\noindent
Let $c \in \mathbb{F}$. Then we have the following,
\begin{eqnarray*}
\langle cu, v \rangle &=& ca_1b_1 + \cdots + ca_nb_n\\
&=& c \langle u, v \rangle, 
\end{eqnarray*}
so the definition satisfies homogeneity in the first slot.\\

\noindent
We have that,
\begin{eqnarray*}
\langle u, v \rangle &=& a_1b_1 + \cdots + a_nb_n\\
\text{and } \langle v, u &=& b_1a_1 + \cdots + b_na_n,
\end{eqnarray*}
so $\langle u, v \rangle = \langle v, u \rangle$, so the definition exhibits symmetry.\\

\noindent
Since $(v_1, \ldots, v_n)$ are eigenvectors of $T$, $\mathcal{M}(T)$ with respect to $(v_1, \ldots, v_n)$ is a diagonal matrix, so $T$ is self adjoint.\\

\noindent
So $U$ has a basis consisting of eigenvectors of $T$ is and only if there is an inner product on $U$ that makes $T$ into a self-adjoint operator.\\
\qed

\problem{3} Suppose $T$ is orthogonal. We have the following for $v \in V$,
\begin{eqnarray*}
||v||^2 &=& \langle v, v \rangle\\
||Tv||^2 &=& \langle Tv, Tv \rangle.
\end{eqnarray*}
Since $T$ is orthogonal, $\langle v, v \rangle = \langle Tv, Tv \rangle$, so $||v||^2 = ||Tv||^2$ or $||v|| = ||Tv||$, so if $T$ is orthogonal, it preserves norms.\\

\noindent
Now suppose that $T$ preserves norms. So for $u, v \in V$, $||u|| = ||Tu||$ and $||v|| = ||Tv||$. Now note that,
\begin{eqnarray*}
||u + v||^2 &=& ||u||^2 + ||v||^2 + 2 \langle u, v \rangle.
\end{eqnarray*}
Also,
\begin{eqnarray*}
||T(u + v)||^2 &=& ||Tu + Tv||\\
&=& ||Tu||^2 + ||Tv||^2 + 2 \langle Tu, Tv \rangle.\\
\end{eqnarray*}
Since $T$ preserves, norms, the above equation is in fact,
$$
||u + v||^2 = ||u||^2 + ||v||^2 + 2\langle Tu, Tv \rangle.
$$
But $||u + v||^2 = ||u||^2 + ||v||^2 + 2 \langle u, v \rangle$, so it must be that $\langle u, v \rangle = \langle Tu, Tv \rangle$, so by definition, $T$ is orthogonal.\\

\noindent
Hence $T$ preserves norms if and only if $T$ is orthogonal.\\
\qed 
\end{document}