\documentclass[12pt]{article}

%%%%%%%%%%%%%%%%%%%%%%%%%%%%%%%%%%%%%%%%%%%%%%%%%%%%%%%%
% Packages go below
%%%%%%%%%%%%%%%%%%%%%%%%%%%%%%%%%%%%%%%%%%%%%%%%%%%%%%%%
%
\usepackage{fullpage}
\usepackage{graphicx}

\usepackage{amssymb}
\usepackage{amsthm}
\usepackage{amsmath}
\usepackage{multicol}
\usepackage{asymptote}
\usepackage{epstopdf}
\DeclareGraphicsRule{.tif}{png}{.png}{`convert #1 `dirname #1`/`basename #1 .tif`.png}

\newcommand{\problem}[1]{\vspace{0.3in} \noindent {\bf Problem #1}}
\newcommand{\solution}[1]{\vspace{0.3in} \noindent\bf Solution #1}

%%%%%%%%%%%%%%%%%%%%%%%%%%%%%%%%%%%%%%%%%%%%%%%%%%%%%%%%
% Assignment specific titles below
%%%%%%%%%%%%%%%%%%%%%%%%%%%%%%%%%%%%%%%%%%%%%%%%%%%%%%%%
%
\title{\bf \large Harvard University \\ Math 25B \\ \vspace{0.2in} Problem Set 9}
\author{ \bf \large Problem set by  Anupa Murali}
\date{\today}                

\begin{document}
\maketitle

\thispagestyle{empty}



\vspace{0.35in}

\centerline{\bf \large Part 3}

\problem{1} 
\begin{itemize}
\item[(a)] Note that $\int_\gamma d\theta = \arctan{(y/x)}$. We are evaluating $\int_\gamma d\theta$ along the ellipse. Note that $\arctan{(y/x)}$ ranges from 0 to $2\pi$ on the ellipse. Hence it is likely that $\int_\gamma d\theta = 2\pi$. Also, note the following,

\begin{eqnarray*}
 \int_\gamma \frac{-ydx + x dy}{x^2 + y^2} &=& \int_0^{2\pi} \frac{ab\sin^2t + ab\cos^2t}{a^2\cos^2t + b^2 \sin^2t}dt \\
 &=& \int_0^{2\pi}  \frac{ab}{a^2\cos^2t + b^2 \sin^2t}dt\\
\Longrightarrow  \int_0^{2\pi}  \frac{1}{a^2\cos^2t + b^2 \sin^2t}dt &=& \frac{2\pi}{ab}.
\end{eqnarray*}
\qed

\item[(b)] We first compute $E,F,G$ as follows,
\begin{eqnarray*}
E &=& \frac{\partial T}{\partial u} \cdot \frac{\partial T}{\partial u}\\
 &=& (\cos\varphi\cos\theta,\cos\varphi\sin\theta,-\sin\varphi) \cdot (\cos\varphi\cos\theta,\cos\varphi\sin\theta,-\sin\varphi) \\
&=& \cos^2\varphi \cos^2\theta + \cos^2\varphi\sin^2\theta + \sin^2\varphi\\
&=& 1\\
\end{eqnarray*}
\begin{eqnarray*}
F&=& \frac{\partial T}{\partial u}\cdot \frac{\partial T}{\partial v}\\
&=& (\cos\varphi\cos\theta,\cos\varphi\sin\theta,-\sin\varphi) \cdot (-\sin \varphi \sin \theta,\sin\varphi \cos \theta,0)\\
&=& -\cos\varphi\sin\varphi\cos\theta\sin\theta + \cos\varphi\sin\varphi\cos\theta\sin\theta + 0\\
&=& 0\\
\end{eqnarray*}
\begin{eqnarray*}
G&=& \frac{\partial T}{\partial v}\cdot \frac{\partial T}{\partial v}\\
&=& (-\sin \varphi \sin \theta,\sin\varphi \cos \theta,0) \cdot (-\sin \varphi \sin \theta,\sin\varphi \cos \theta,0)\\
&=& \sin^2\varphi \sin^2\theta + \sin^2\varphi \cos^2\theta\\
&=& \sin^2\varphi.\\
\end{eqnarray*}
Hence plugging into the formula derived from part (a), we have,
\begin{eqnarray*}
s(\gamma) &=& \int_a^b \left[\left(\frac{d\varphi}{dt}\right)^2 + \sin^2\varphi\left(\frac{d\theta}{dt}\right)^2\right]^{1/2}dt.
\end{eqnarray*}
\qed
\end{itemize}
\problem{2} 
Define $F$ such that $\nabla F=(f,g)$. So we have the following,
\begin{eqnarray*}
\int_\gamma \omega &=& \int_\gamma f(x)dx + g(y)dy\\ 
&=& \int_{t_1}^{t_2} f(\gamma(t))\gamma_1'(t) + g(\gamma(t))\gamma_2'(t)dt\\
&=& \int_{t_1}^{t_2} \nabla F (\gamma(t)) \cdot \gamma'(t) dt\\
&=& F(\gamma(t_2)) - F(\gamma(t_1)).
\end{eqnarray*}
However, since $\gamma$ is closed, $\gamma(t_1) = \gamma(t_2)$, so $\int_\gamma \omega = 0$.\\
\qed


\problem{3} We compute it as follows,
\begin{eqnarray*}
\int_\gamma \omega &=& \int_0^{2\pi}\frac{(\cos t - \sin t)(-\sin t) + (\cos t + \sin t) \cos t}{\cos^2t + \sin^2t}dt\\
&=& \int_0^{2\pi}dt\\
&=& \boxed{2\pi}.
\end{eqnarray*}
\end{document}