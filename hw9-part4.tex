\documentclass[12pt]{article}

%%%%%%%%%%%%%%%%%%%%%%%%%%%%%%%%%%%%%%%%%%%%%%%%%%%%%%%%
% Packages go below
%%%%%%%%%%%%%%%%%%%%%%%%%%%%%%%%%%%%%%%%%%%%%%%%%%%%%%%%
%
\usepackage{fullpage}
\usepackage{graphicx}

\usepackage{amssymb}
\usepackage{amsthm}
\usepackage{amsmath}
\usepackage{multicol}
\usepackage{asymptote}
\usepackage{epstopdf}
\DeclareGraphicsRule{.tif}{png}{.png}{`convert #1 `dirname #1`/`basename #1 .tif`.png}

\newcommand{\problem}[1]{\vspace{0.3in} \noindent {\bf Problem #1}}
\newcommand{\solution}[1]{\vspace{0.3in} \noindent\bf Solution #1}

%%%%%%%%%%%%%%%%%%%%%%%%%%%%%%%%%%%%%%%%%%%%%%%%%%%%%%%%
% Assignment specific titles below
%%%%%%%%%%%%%%%%%%%%%%%%%%%%%%%%%%%%%%%%%%%%%%%%%%%%%%%%
%
\title{\bf \large Harvard University \\ Math 25B \\ \vspace{0.2in} Problem Set 9}
\author{ \bf \large Problem set by  Anupa Murali}
\date{\today}                

\begin{document}
\maketitle

\thispagestyle{empty}



\vspace{0.35in}

\centerline{\bf \large Part 4}
\problem{1} The function could be $f(x,y,z) = xy + z^2$. Using Theorem 1.5, we have the following,
\begin{eqnarray*}
\int_\gamma \omega &=& \int_a^b df_\gamma(\gamma'(t))dt\\
&=& f(\gamma(b)) - f(\gamma(a))\\
&=& f(0,1,-1) - f(1,0,1)\\
&=& \boxed{0}.
\end{eqnarray*}


\problem{2} 
\begin{itemize}
\item[(a)] Using Green's Theorem, we have,
\begin{eqnarray*}
 \int_C (3x^2-y)dx + (x+4y^3)dy &=&\int_D\int 2 dxdy\\
 &=& \int_0^1\int_0^{\sqrt{1-y^2}}2dxdy\\
 &=& \int_0^12\sqrt{1-y^2}dy\\
 &=& (x\sqrt{1-x^2} + \arcsin x)\Big |_0^1\\
 &=& \boxed{\pi/2}.
 \end{eqnarray*}
 \item[(b)] Again using Green's Theorem, we have,
 \begin{eqnarray*}
 \int_0^1 2xdx &=& x^2 \Big |_0^1\\
 &=& \boxed{1}.
 \end{eqnarray*}
\end{itemize}

\problem{3} Using the parametrization $(x,y) = (a\cos x,  b\sin x)$, we have,
\begin{eqnarray*}
\frac{1}{2} \int -ydx + x dy &=& \frac{1}{2} \int_0^{2\pi} ab\sin^2x + ab\cos^2x dx\\
&=& \frac{1}{2}\int_0^{2\pi}ab dx\\
&=& \frac{1}{2} ab \Big |_0^{2\pi}\\
&=& \boxed{ \pi ab}.
\end{eqnarray*}


\end{document}